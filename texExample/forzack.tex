\documentclass{report}
\begin{document}
Example in Washington/Trapp, page 189-191.\\

Factorization of $n=3837523$  using the  remainders of some squares $\pmod{n}$.
Note, in the book there is a 7th  column for the prime $17$, which is 
all zeros and so we can safely delete it and we only consider
the seven primes $2,3,5,7,11,13,19$ to factor the remainders.  The following matrix
gives the exponent vectors for the remainders of the selected  squares $\pmod{n}$.
The integers, whose squares we compute, are  in the first column.
They are all of  the form $[\sqrt{i n} +j]$ as discussed in the book and $i,j$
are listed in column $2,3$. \\
\[
F := \left(\begin{array}{r|rr|rrrrrrr|r}
n=3837523   & i & j & 2 & 3 & 5 & 7 & 11 & 13 & 19 & {\rm remainder} \\ \hline
 9398^2 \pmod{n} & 23 & 4  & 0 & 0 & 5 & 0 & 0 & 0 &  1 & 59375 \\
 19095^2 \pmod{n} & 95 & 2  & 2 & 0 & 1 & 0 & 1 & 1 &  1 & 54340 \\
 1964^2 \pmod{n} & 1 & 6  & 0 & 2 & 0 & 0 & 0 & 3 &  0 &  19773 \\
 17078^2 \pmod{n} & 76 & 1  & 6 & 2 & 0 & 0 & 1 & 0 &  0 & 6336 \\
 8077^2 \pmod{n} & 17 & 1  & 1 & 0 & 0 & 0 & 0 & 0 &  1 &  38 \\
 3397^2 \pmod{n} & 3 & 4  & 5 & 0 & 1 & 0 & 0 & 2 &  0 &  27040 \\
 14262^2 \pmod{n} & 53 & 1  & 0 & 0 & 2 & 2 & 0 & 1 &  0 & 15925 \\ \hline
 \end{array}\right) 
\]

So for example $9398^2 \pmod{n} = 59375 = 2^0 \cdot 3^0 \cdot 5^5 \cdot 7^0
\cdot 11^0 \cdot 13^0 \cdot 19^1$. \\
To find a product of the remainders, which is a square, we look  compute
the following matrix $A := F \pmod{2}$ by replacing an even number in an exponent vector by
$0$ and an odd number in an exponent vector by $1$:
\[
 A := \left(\begin{array}{rrrrrrr}
  0 & 0 & 1 & 0 & 0 & 0 &  1 \\
  0 & 0 & 1 & 0 & 1 & 1 &  1 \\
  0 & 0 & 0 & 0 & 0 & 1 &  0 \\
  0 & 0 & 0 & 0 & 1 & 0 &  0 \\
  1 & 0 & 0 & 0 & 0 & 0 &  1 \\
  1 & 0 & 1 & 0 & 0 & 0 &  0 \\
  0 & 0 & 0 & 0 & 0 & 1 &  0 \\
\end{array}\right)
\]
A basis for left nullspace of $A$  is given by the following three vectors
of the matrix $N$
using standard row reduction:( note the last column of $A$ is  the sum of the first and  the forth column of $A$).
\[
N:= \begin{array}{rrrrrrr}
  1 & 0 & 0 & 0 & 1 & 1 & 0 \\
  1 & 1 & 1 & 1 & 0 & 0 & 0 \\
  0 & 0 & 1 & 0 & 0 & 0 & 1 \\ 
 \end{array}
\]
Taking the sum of the corresponding exponent vectors in $F$ .

\[
 \begin{array}{rrrrrrr}
   2 & 3 & 5 & 7 & 11 & 13 & 19 \\ \hline
6 & 0 & 6 & 0 & 0 & 2 & 2 \\
8 & 4 & 6 & 0 & 2 & 4 & 2 \\ 
0 & 2 & 2 & 2 & 0 & 4 & 0 \\
\end{array}
\]
and dividing the entries by $2$ to get the exponent vector of a square root
we get:

\[
 \begin{array}{rrrrrrr}
   2 & 3 & 5 & 7 & 11 & 13 & 19 \\ \hline
3 & 0 & 3 & 0 & 0 & 1 & 1 \\
4 & 2 & 3 & 0 & 1 & 2 & 1  \\
0 & 1 & 1 & 1 & 0 & 2 & 0 \\
\end{array}
\]

This tells us using the first vector of $N$ and  the corresponding
numbers $9398, 8077$ and $3397$:
\[
 \begin{array}{cc}
  (9398 \cdot 8077 \cdot 3397)^2 \equiv (2^3 \cdot 3^0 \cdot 5^3 \cdot 7^0
  \cdot 11^0 \cdot 13^1 \cdot 19^1)^2 \pmod{n}
 \end{array}
 \]
 Note that on the left hand side  $X:=9398\cdot 8077 \cdot 3397 \pmod{n} = 3590523$
 and on the right hand side $Y:=2^3 \cdot 3^0 \cdot 5^3 \cdot 7^0 \cdot 11^0 \cdot 13^1 \cdot 19^1 \pmod{n} =  247000$. So we get:
 \[ 
  3590523^2 \equiv 247000^2 \pmod{n}
 \]
 and we can test ${\rm gcd}(X-Y,n)$. It is $n$.

 Applying this recipe to  the second vector of $N$,   the left hand side is

\[
 9398\cdot 19095 \cdot 1964 \cdot 17078 \pmod{n} = 2230387
\]

and the right hand side gives
\[
 635778000 \pmod{n} =   2586705,
\]
so we get
\[
 2230387^2 \equiv 2586705^2 \pmod{n}
\]
and ${\rm gcd}(2230387 - 2586705, n) = 1093$.
\end{document}
